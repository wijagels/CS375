\documentclass[a4paper,12pt]{article}

\usepackage[english]{babel}
\usepackage[T1]{fontenc}
\usepackage[utf8]{inputenc}
\usepackage{amsmath}
\usepackage{amsthm}
\usepackage{forest}
\usepackage{graphicx}
\usepackage{xcolor}
\usepackage[colorinlistoftodos]{todonotes}
\usepackage{enumitem}
\usepackage{DejaVuSansMono}
\usepackage{listings}
\lstset{basicstyle=\footnotesize\ttfamily,breaklines=true}
\lstset{frame=tb,language=python,numbers=left,showstringspaces=false}
\renewcommand{\lstlistingname}{Code Block}

\usepackage{geometry}
\geometry{total={210mm,297mm},
left=20mm, right=20mm,
bindingoffset=0mm,
top=20mm, bottom=20mm}

\usepackage[
  pdftitle={Assignment 2},
  pdfauthor={William Jagels},
  colorlinks=true,linkcolor=blue,urlcolor=blue,citecolor=blue,bookmarks=true,
bookmarksopenlevel=2]{hyperref}

\usepackage{titlesec}
\titlelabel{\thetitle.\quad}

\def\code#1{\texttt{#1}}

\title{Assignment 2}

\author{William Jagels}

\date{\today}

\begin{document}
\maketitle

\section{Integer Powers}
\subsection{Function}
\begin{lstlisting}
def power(x, y):
    y = int(y)
    if y == 1:
        return x
    r = power(x, y/2)
    if y % 2 == 0:
        return r * r
    return x * r * r
\end{lstlisting}
\subsection{Recurrence Equations}
Let $n = y$
$$T(n) = T(n/2) + c_2$$

$$T(n) = c_2 \log_2{n} + c_1$$

For even case, $c_2 = 1$ because of one multiplication per recursion.
For the odd case, $c_2 = 2$ because of the additional multiplication.
$c_1 = 0$ because the base case of $T(1)$ requires zero multiplications.

\begin{equation}
T(n) =
\begin{cases}
  \log_2{n}, & $if\ n is even$\\
  2\log_2{n}, & $if\ n is odd$
\end{cases}
\end{equation}

\section{Direct vs Recursive}
Recursive solution

$$T(n) = 2T(n/2) + n\log_2 n + \log_2 n$$

$$f(n) = n\log_2 n + \log_2 n = O(n \log n)$$

$$T(n) = n^{\log_2 2} * f(n)$$

$$T(n) = O(n \log_2^2 n)$$

Therefore, recursive solution is better.

\section{Flood fill}
\begin{lstlisting}
def flood(color, x, y, target, N):
    if x > N or y > N:
        return
    if color[x, y] = BLACK:
        return
    if color[x, y] != target:
        color[x, y] = target
    else:
        return
    flood(color, x + 1, y, target)
    flood(color, x - 1, y, target)
    flood(color, x, y + 1, target)
    flood(color, x, y - 1, target)
\end{lstlisting}
x and y are the coordinates of an arbitrary point within the region to be filled, and target is the color desired.


\section{Index equals element}
\subsection{Algorithm}
\begin{lstlisting}
def indexeq(A):
    return indexeq_h(A, 0, len(A)-1)

def indexeq_h(A, l, u):
    if l == u:
        return A[l] == l
    if u - l == 1:
        return A[l] == l or A[u] == u
    mid = l + int(u/2)
    if A[mid] == mid:
        return True
    if A[mid] > mid:
        return indexeq_h(A, mid, u)
    return indexeq_h(A, l, mid)
\end{lstlisting}

\subsection{Time complexity}
$\Theta (\log_{2} n)$ because at every step, the search space is halved due to
the assumption that the list is sorted.
\begin{equation}
T(N) \le \begin{cases}
  N, & if\ N \le 2\\
  \lg(N), & if\ N > 2
\end{cases}
\end{equation}
In cases where the middle element of the list satisfies the condition, the runtime will be $T(1)$.
At worst, the algorithm will take $T(\log_{2}{N})$ steps.


\section{Graphing}
\subsection{Pseudocode}
\begin{lstlisting}
def makegraph(x, y, r):
    if r == 0:
        return
    DrawSquare(x, y, r)
    n = r\2
    makegraph(x + r, y + r, n) # Q1
    makegraph(x - r, y + r, n) # Q2
    makegraph(x - r, y - r, n) # Q3
    makegraph(x + r, y - r, n) # Q4
    return
\end{lstlisting}

\subsection{Recurrence equation}
$$T(r) = 4 T(r/2) + 1$$

$$T(r) = \frac{1}{12}((3 c_1 + 4) r^2 - 4)$$

$c_1 = 4$ because $T(1) = 1$

$$T(r) = \frac{1}{12}((3(4) + 4) r^2 - 4)$$

\subsection{Recursion Tree}
\begin{forest}
for tree={
  parent anchor=south,
  s sep=0pt
}
[$T(n)$
  [$T(\frac{n}{2})$
    [$T(\frac{n}{4})$]
    [$T(\frac{n}{4})$]
    [$T(\frac{n}{4})$]
    [$T(\frac{n}{4})$]
  ]
  [$T(\frac{n}{2})$
    [$T(\frac{n}{4})$]
    [$T(\frac{n}{4})$]
    [$T(\frac{n}{4})$]
    [$T(\frac{n}{4})$]
  ]
  [$T(\frac{n}{2})$
    [$T(\frac{n}{4})$]
    [$T(\frac{n}{4})$]
    [$T(\frac{n}{4})$]
    [$T(\frac{n}{4})$]
  ]
  [$T(\frac{n}{2})$
    [$T(\frac{n}{4})$]
    [$T(\frac{n}{4})$]
    [$T(\frac{n}{4})$]
    [$T(\frac{n}{4})$]
  ]
]
\end{forest}


\section{Majority element}
\begin{lstlisting}
def freq(A, candidate):
    c = 0
    for e in A:
        if e == candidate:
            c += 1
    return c

def maj(A):
    if len(A) == 1:
        return A[0]
    candidate_l = maj(A[:len(A)//2])
    candidate_r = maj(A[len(A)//2:])
    if candidate_l == candidate_r:
        # Best case
        return candidate_l
    if freq(A, candidate_l) > len(A)//2:
        return candidate_l
    if freq(A, candidate_r) > len(A)//2:
        # Worst case
        return candidate_r
\end{lstlisting}


\section{MinMax}
\subsection{Recurrence equation}
$$T(n) = T(n/2) + T(n/2) + 2$$
Each recursion step performs 2 comparisons, and calls 2 other steps with the left and right halves.
\subsection{Recursion tree}
\begin{forest}
for tree={
  parent anchor=south,
  s sep=2pt
}
[$T(n) + c_2$
  [$T(\frac{n}{2}) + c_2$
    [$T(\frac{n}{4}) + c_2$]
    [$T(\frac{n}{4}) + c_2$]
  ]
  [$T(\frac{n}{2}) + c_2$
    [$T(\frac{n}{4}) + c_2$]
    [$T(\frac{n}{4}) + c_2$]
  ]
]
\end{forest}
$$T(n) = \frac{c_1n}{2} + c_2 (n-1)$$
$c_2 = 2$ because of the two comparisons per recursive step.
Therefore, $c_1 = 4$

\section{Master Method}
\subsection{}
$$T(n) = 9 T(\frac{n}{3}) + n^2 + 4$$

$$f(n) = n^2 + 4 = O(n^2)$$

$$T(n) = n^{\log_3 9} * f(n)$$

$$h(n) = \frac{n^2 + 4}{n^{\log_3 9}}$$

$$h(n) = \frac{n^2 + 4}{n^2}$$

Same order, so

$$T(n) = O(n^2\lg n)$$
\subsection{}
$$T(n) = 6 T(\frac{n}{2}) + n^2 - 2$$

$$f(n) = n^2 - 2 = O(n^2)$$

$$T(n) = n^{\log_2 6} * f(n)$$

$\log_2 6 > 2$ so $O(1)$
$$T(n) = O(n^{\log_2 6}) * O(1)$$

$$T(n) = O(n^{\log_2 6})$$

\subsection{}
$$T(n) = 4 T(\frac{n}{2}) + n^3 + 7$$

$$f(n) = n^3 + 7 = O(n^3)$$

$$T(n) = n^{\log_2 4} * f(n)$$

$$T(n) = n^2 * f(n)$$

Lower order than $f(n)$, so

$$T(n) = O(1) * O(n^3)$$

$$T(n) = O(n^3)$$

\section{Loop invariant}
\begin{proof}
$$A{[i]} = \sqrt{A{[i-1]}^2 + X{[i]}^2}$$

Base case: $n = 1: A{[0]} = X{[0]}, A{[1]} = \sqrt{X{[0]}^2 + X{[1]}^2}$

Assume that at $i$th iteration, $A[i] = \sqrt{\sum_{k=0}^{i}{X[k]^2}}$

$$A[i]^2 = \sum_{k=0}^i (X[k]^2)$$

Then, at the $i+1$th iteration, we have:

$$A[i + 1] = \sqrt{\sum_{k=0}^{i+1}{X[k]^2}}$$

$$A[i + 1] = \sqrt{\sum_{k=0}^{i}{X[k]^2} + X[i+1]^2}$$

$$A[i + 1] = \sqrt{\sum_{k=0}^{i}{X[k]^2} + X[i+1]^2}$$

$$A[i + 1] = \sqrt{A[i]^2 + X[i+1]^2}$$
\end{proof}

\section{Recurrence equation}
$T(n)=8T(n-1)-21T(n-2)+18T(n-3)$ for $n>2$

$$T(n) = C_1*r_1^n + C_2*r_2^n + C_3*n*r_3^n$$

$$r_1 = 2,\ r_2 = 3,\ r_3 = 3$$

$$T(n) = C_1 * 2^n + C_2 * 3^n + C_3 * n * 3^n$$

$T(0) = 0$ therefore, $C_1 + C_2 = 0$.

$T(1) = 1$ therefore, $C_3 = \frac{C_1}{3} + \frac{1}{3}$.

$T(2) = 2$ therefore, $C_1 = -4$.

Then, $C_2 = 4$ and $C_3 = \frac{-4}{3} + \frac{1}{3} = -1$.

$$T(n) = -4 * 2^n + 4 * 3^n - 1 * n * 3^n$$


\end{document}
